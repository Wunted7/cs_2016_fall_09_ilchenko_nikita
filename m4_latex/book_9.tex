\documentclass[fleqn,12 pt]{book}

\usepackage[english]{babel}
\usepackage[utf8x]{inputenc}
\usepackage[T1]{fontenc}

\usepackage[a4paper,top=3cm,bottom=2cm,left=3.5cm,right=3.5cm,marginparwidth=1.75cm,]{geometry}

\usepackage{amssymb}
\usepackage{amsmath}
\usepackage{graphicx}
\usepackage[colorlinks=true, allcolors=blue]{hyperref}
\usepackage{color}

\pagestyle{empty}
\definecolor{light-gray}{rgb}{0.8,0.8,0.8}

\setcounter{chapter}{2}
\setcounter{section}{4}
\newcounter{pro1}
\setcounter{pro1}{20}
\newcommand{\pro}{\par\addtocounter{pro1}{1}%
\textbf{Problem \arabic{chapter}.\arabic{pro1} }\quad}
\setcounter{equation}{29}
\addto\captionsenglish{\renewcommand{\chaptername}{}}

\begin{document}
\noindent
To test this prediction, I constructed the small and large cones described on~page~21,~held one in each hand above my head, and let them fall.
Their 2~m all lasted roughly 2~s~, and they landed within 0.1 s of one another.
Cheap experiment and cheap theory agree!\\
\par
\noindent
\colorbox{light-gray}{
\begin{minipage}{\textwidth}
{\footnotesize
\pro
\textbf{Home experiment of a small versus a large cone}\\
The cone home experiment yourself (page~21).\\
\pro
\textbf{{\scriptsize Home experiment of four stacked cones versus one cone}}\\
Predict the ratio
\begin{equation}
\frac{\text{terminal velocity of four small cones stacked inside each other}}{\text{terminal velocity of one small cone}}.
\end{equation}
Test your prediction. Can you find a method not requiring timing equipment?\\
\pro
\textbf{Estimating the terminal speed}\\
\noindent
Estimate or look up the areal density of paper; predict the cones’ terminal speed;
and then compare that prediction to the result of the home experiment.}
\end{minipage}}

\section{Summary and further problems}
\noindent
A  correct  solution  works  in  all  cases, including  the  easy  ones.   There-fore, check any proposed formula in the easy cases, and guess formulas by constructing expressions that pass all easy-cases tests. To apply and extend these ideas, try the following problems and see the concise and instructive book by Cipra [10].\\
\par
\noindent
\colorbox{light-gray}{
\begin{minipage}{\textwidth}
{\footnotesize
\pro
\textbf{Fencepost errors}\\
A garden has 10~m of horizontal fencing that you would like to divide into 1~m segments by using vertical posts. Do you need 10 or 11 vertical posts (including the posts needed at the ends)?\\
\pro
\textbf{Odd sum}\\
Here is the sum of the first $n$ odd integers:
\begin{equation}
S_n=\underbrace{1+3+5+\cdots+l_n}_\textnormal{n terms}
\end{equation}
\noindent
\begin{itemize}
\item[a.]\ Does the last term $\textnormal{l}_n$ equal $2\textnormal{n}+1$ or $2\textnormal{n}-1$?
\item[b.]\ Use easy cases to guess $\textnormal{S}_n$ (as a function of n).
\end{itemize}
\par An alternative solution is discussed in Section 4.1.}
\end{minipage}}
\newpage
\colorbox{light-gray}{
\begin{minipage}{\textwidth}
{\small
\pro
\textbf{Free fall with initial velocity}\\
\noindent
The ball in Section 1.2 was released from rest. Now imagine that it is given an initial velocity $v_0$ (where positive $v_0$ means an upward throw). Guess the impact velocity $v_i$.\\
Then solve the free-fall differential equation to find the exact $v_i$~, and compare the exact result to your guess.\\
\pro
\textbf{Low Reynolds number}\\
In the limit $\mbox{Re}\ll 1$~, guess the form of f in
\begin{equation}
\frac{\mbox{F}}{\rho v^2 \mbox{r}^2}=\mbox{f}\left(\frac{\mbox{r}v}{v}\right)
\end{equation}\\
\noindent
The result, when combined with the correct dimensionless constant, is known
as Stokes drag [12].\\
\pro
\textbf{Range formula}\\
\hangindent=-3cm\noindent
How far does a rock travel horizontally (no air resistance)?
Use dimensions and easy cases to guess a formula for the range R as a function of the launch velocity $v$~, the launch angle $\theta$~, and the gravitational acceleration g.\\
\pro
\textbf{Spring equation}\\
\noindent
The angular frequency of ideal mass-spring system (Section 3.4.2) is $\sqrt{k/m}$~, where k is the spring constant and m is the mass.
This expression has the spring constant k in the numerator.
Use extreme cases of k or m to decide whether that placement is correct.\\
\pro
\textbf{Taping the cone templates}\\
\noindent
The tape mark on the large cone template (page~21) is twice as wide as the tape
mark on the small cone template.
In other words, if the tape on the large cone is, say, 6\,mm wide, the tape on the small cone should be 3\,mm wide.
Why?}
\end{minipage}}
\newpage
\chapter{Lumping}
\par\bigskip
\colorbox{light-gray}{
\begin{minipage}{\textwidth}
3.1\quad Estimating populations: How many babies?\\
3.2\quad Estimating integrals\\
3.3\quad Estimating derivatives\\
3.4\quad Analyzing differential equations: The spring–mass system\\
3.5\quad Predicting the period of a pendulum\\
3.6\quad \textit{Summary and further problems}
\end{minipage}}\\\\
\par\bigskip
\noindent
Where will an orbiting planet be 6 months from now?
To predict its new location, we cannot simply multiply the 6 months by the planet’s current velocity, for its velocity constantly varies.
Such calculations are the reason that calculus was invented.
Its fundamental idea is to divide the time into tiny intervals over which the velocity is constant, to multiply each tiny time by the corresponding velocity to compute a tiny distance, and then to add the tiny distances.

\noindent
Amazingly, this computation can often be done exactly, even when the intervals have infinitesimal width and are therefore infinite in number.
However, the symbolic manipulations can be lengthy and, worse, are often rendered impossible by even small changes to the problem.
Using calculus methods, for example, we can exactly calculate the area under the Gaussian $e^{-x^2}$ between $x=0$ and $\infty$; yet if either limit is any value except zero or infinity, an exact calculation becomes impossible.

\noindent
In contrast, approximate methods are robust: They almost always provide a reasonable answer.
And the least accurate but most robust method is lumping.
Instead of dividing a changing process into many tiny pieces, group or lump it into one or two pieces.
This simple approximation and its advantages are illustrated using examples ranging from demographics (Section 3.1) to nonlinear differential equations (Section 3.5).\newpage
\section{{\small Estimating populations: How many babies?}}

\noindent
The first example is to estimate the number of babies in the United States.
For definiteness, call a child a baby until he or she turns 2 years old.
An exact calculation requires the birth dates of every person in the United States.
This, or closely similar, information is collected once every decade by the US Censu~Bureau.
\par\bigskip
\hangindent=-4.5cm\noindent\hangafter=0
As an approximation to this voluminous data, the Census Bureau [47] publishes the number of people at each age.
The data for $1991$ is a set of points lying on a wiggly line $N(t)$, where $t$ is age.
Then
\begin{flushleft}
\begin{equation}
N_\text{babies}=\int^{2 \text{yr}}_0 N(t)dt
\end{equation}
\end{flushleft}
\colorbox{light-gray}{
\begin{minipage}{\textwidth}
{\footnotesize
\pro
\textbf{Dimensions of the vertical axis}\\
Why is the vertical axis labeled in units of people per year rather than in units
of people?  Equivalently, why does the axis have dimensions of $T^{-1}$?}
\end{minipage}}\\

\noindent
This method has several problems.
First, it depends on the huge resources of the US Census Bureau, so it is not usable on a desert island for back-of-the-envelope calculations.
Second, it requires integrating a curve with no analytic form, so the integration must be done numerically.
Third, the integral is of data specific to this problem, whereas mathematics should be about generality.
An exact integration, in short, provides little insight and has minimal transfer value.
Instead of integrating the population curve exactly, approximate it---lump the curve into one rectangle.\\
\begin{flushleft}
$\blacktriangleright$ \textit{What are the height and width of this rectangle?}
\end{flushleft}

\noindent
The rectangle’s width is a time, and a plausible time related to populations is the life expectancy.
It is roughly $80$ years, so make $80$ years the width by pretending that everyone dies abruptly on his or her $80$\,th birthday.
The rectangle’s height can be computed from the rectangle’s area, which
is the US population---conveniently $300$ million in $2008$.
Therefore,
\begin{equation}
\text{height}=\frac{\text{area}}{\text{width}}\thicksim\frac{3 \times 10^8}{75 \text{yr}}
\end{equation}
$\blacktriangleright$ \textit{Why did the life expectancy drop from $80$ to $75$ years?}\newpage
\end{document}
