\documentclass{article}

\usepackage[english]{babel}
\usepackage[utf8x]{inputenc}
\usepackage[T1]{fontenc}

\usepackage[a4paper,top=3cm,bottom=2cm,left=4.5cm,right=4.5cm,marginparwidth=1.75cm]{geometry}

\usepackage{amsmath}
\usepackage{graphicx}
\usepackage[colorlinks=true, allcolors=blue]{hyperref}
\usepackage{color}

\pagestyle{empty}

\setcounter{section}{2}
\setcounter{subsection}{4}
\newcounter{pro1}
\setcounter{pro1}{20}
\newcommand{\pro}{\par\addtocounter{pro1}{1}%
\textbf{Problem \arabic{section}.\arabic{pro1} }\quad}
\newcounter{equa}
\setcounter{equa}{30}

\begin{document}
To test this prediction, I constructed the small and large cones described on~page~21,~held one in each hand above my head, and let them fall.
Their 2~m all lasted roughly 2~s~, and they landed within 0.1 s of one another.
Cheap experiment and cheap theory agree!\\
%\definecolor{light-gray}{gray}{0.7}
%\colorbox{light-gray}{
\pro
\textbf{Home experiment of a small versus a large cone}\\
The cone home experiment yourself (page~21).\\
\pro
\textbf{Home experiment of four stacked cones versus one cone}\\
Predict the ratio
\[\frac{\text{terminal velocity of four small cones stacked inside each other}}{\text{terminal velocity of one small cone}}. \eqno(2.30)\]
Test your prediction. Can you find a method not requiring timing equipment?\\
\pro
\textbf{Estimating the terminal speed}\\
Estimate or look up the areal density of paper; predict the cones’ terminal speed;
and then compare that prediction to the result of the home experiment.\subsection{Summary and further problems}
A  correct  solution  works  in  all  cases,  including  the  easy  ones.   There-fore, check any proposed formula in the easy cases, and guess formulas
by constructing expressions that pass all easy-cases tests. To apply and
extend these ideas, try the following problems and see the concise and
instructive book by Cipra [10].\\
\pro
\textbf{Fencepost errors}\\
A garden has 10~m of horizontal fencing that you would like to divide into 1~m segments by using vertical posts. Do you need 10 or 11 vertical posts (including the posts needed at the ends)?\\
\pro
\textbf{Odd sum}\\
Here is the sum of the first $n$ odd integers:
\addtocounter{equa}{1}
\[S_n=\underbrace{1+3+5+\cdots+l_n}_\textnormal{n terms} \eqno(2.31)\]
\begin{itemize}
\item[a.]\ Does the last term $\textnormal{l}_n$ equal $2\textnormal{n}+1$ or $2\textnormal{n}-1$?
\item[b.]\ Use easy cases to guess $\textnormal{S}_n$ (as a function of n).
\end{itemize}
\par An alternative solution is discussed in Section 4.1.\newpage
\pro
\textbf{Free fall with initial velocity}\\
The ball in Section 1.2 was released from rest. Now imagine that it is given an
initial velocity $v_0$ (where positive $v_0$ means an upward throw). Guess the impact velocity $v_i$.\\
Then solve the free-fall differential equation to find the exact $v_i$~, and compare the exact result to your guess.\\
\pro
\textbf{Low Reynolds number}\\
In the limit $\mbox{Re}\ll 1$~, guess the form of f in
\[
\frac{\mbox{F}}{\rho v^2 \mbox{r}^2}=\mbox{f}\left(\frac{\mbox{r}v}{v}\right)\eqno(2.32) 
\]\\
The result, when combined with the correct dimensionless constant, is known
as Stokes drag [12].\\
\pro
\textbf{Range formula}\\
\hangindent=-3cm
How far does a rock travel horizontally (no air resistance)?
Use dimensions and easy cases to guess a formula for the
range R as a function of the launch velocity $v$~, the launch angle $\theta$~, and the gravitational acceleration g.\\
\pro
\textbf{Spring equation}\\
The angular frequency of ideal mass-spring system (Section 3.4.2) is $\sqrt{k/m}$~, where k is the spring constant and m is the mass.
This expression has the spring constant k in the numerator.
Use extreme cases of k or m to decide whether that placement is correct.\\
\pro
\textbf{Taping the cone templates}\\
The tape mark on the large cone template (page~21) is twice as wide as the tape
mark on the small cone template.
In other words, if the tape on the large cone is, say, 6\,mm wide, the tape on the small cone should be 3\,mm wide.
Why?\newpage
\section{Lumping}




\end{document}